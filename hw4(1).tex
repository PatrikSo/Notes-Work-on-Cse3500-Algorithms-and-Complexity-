\documentclass{article}

\usepackage{fancyhdr}
\usepackage{extramarks}
\usepackage{amsmath}
\usepackage{amsthm}
\usepackage{amsfonts}
\usepackage{tikz}
\usepackage[plain]{algorithm}
\usepackage{algpseudocode}

\usetikzlibrary{automata,positioning}


\usepackage{listings}
\usepackage{color}
\usepackage{textcomp}

\newtheorem{theorem}{Theorem}
\newtheorem{property}{Property}
\newtheorem{lemma}{Lemma}
\newtheorem{fact}{Fact}
\newtheorem{definition}{Definition}
\newtheorem{corollary}{Corollary}
\newtheorem{convention}{Convention}
\newtheorem{remark}{Remark}
\newtheorem{observation}{Observation}
\newtheorem{conjecture}{Conjecture}


\definecolor{pblue}{rgb}{0.13,0.13,1}
\definecolor{pgreen}{rgb}{0,0.5,0}
\definecolor{pred}{rgb}{0.9,0,0}
\definecolor{pgrey}{rgb}{0.46,0.45,0.48}




\lstdefinelanguage{bash}{
  keywords={sudo, apt-get, ifconfig, mosquitto_pub, mosquitto_sub, ping},
  basicstyle=\small,
  alsoletter=?!-,
  alsodigit=\$\%&*+./:<=>@^_~,
  sensitive=false,
  morecomment=[l]{\#},
  morecomment=[s]{/*}{*/},
  morestring=[b]'',
  basicstyle=\small\ttfamily,
  keywordstyle=\bf\ttfamily\color[rgb]{0,.3,.7},
  commentstyle=\color[rgb]{0.133,0.545,0.133},                                                                            
  stringstyle={\color[rgb]{0.75,0.49,0.07}},
  upquote=true,
  breaklines=true,
  breakatwhitespace=true,
  literate=*{`}{{`}}{1},
  keywords=[2]{myRed},
  keywordstyle=[2]\color{red}
}

\lstset{language=Python,
  keywords={nx,plt,np,mp,pg},
  showspaces=false,
  showtabs=false,
  breaklines=true,
  showstringspaces=false,
  breakatwhitespace=true,
  commentstyle=\color{pgreen},
  keywordstyle=\color{red},
  stringstyle=\color{pred},
  basicstyle=\ttfamily,
  moredelim=[il][\textcolor{pgrey}]{$$},
  moredelim=[is][\textcolor{pgrey}]{\%\%}{\%\%}
}


%
% Basic Document Settings
%

\topmargin=-0.45in
\evensidemargin=0in
\oddsidemargin=0in
\textwidth=6.5in
\textheight=9.0in
\headsep=0.25in

\linespread{1.1}

\pagestyle{fancy}
\lhead{\hmwkAuthorName}
\chead{\hmwkClass\ (\hmwkClassInstructor\ \hmwkClassTime): \hmwkTitle}
\rhead{\firstxmark}
\lfoot{\lastxmark}
\cfoot{\thepage}

\renewcommand\headrulewidth{0.4pt}
\renewcommand\footrulewidth{0.4pt}

\setlength\parindent{0pt}

%
% Create Problem Sections
%

\newcommand{\enterProblemHeader}[1]{
    \nobreak\extramarks{}{Problem \arabic{#1} continued on next page\ldots}\nobreak{}
    \nobreak\extramarks{Problem \arabic{#1} (continued)}{Problem \arabic{#1} continued on next page\ldots}\nobreak{}
}

\newcommand{\exitProblemHeader}[1]{
    \nobreak\extramarks{Problem \arabic{#1} (continued)}{Problem \arabic{#1} continued on next page\ldots}\nobreak{}
    \stepcounter{#1}
    \nobreak\extramarks{Problem \arabic{#1}}{}\nobreak{}
}

\setcounter{secnumdepth}{0}
\newcounter{partCounter}
\newcounter{homeworkProblemCounter}
\setcounter{homeworkProblemCounter}{1}
\nobreak\extramarks{Problem \arabic{homeworkProblemCounter}}{}\nobreak{}

%
% Homework Problem Environment
%
% This environment takes an optional argument. When given, it will adjust the
% problem counter. This is useful for when the problems given for your
% assignment aren't sequential. See the last 3 problems of this template for an
% example.
%
\newenvironment{homeworkProblem}[1][-1]{
    \ifnum#1>0
        \setcounter{homeworkProblemCounter}{#1}
    \fi
    \section{Problem \arabic{homeworkProblemCounter}}
    \setcounter{partCounter}{1}
    \enterProblemHeader{homeworkProblemCounter}
}{
    \exitProblemHeader{homeworkProblemCounter}
}

%
% Homework Details
%   - Title
%   - Due date
%   - Class
%   - Section/Time
%   - Instructor
%   - Author
%

\newcommand{\hmwkTitle}{Homework\ \#4}
\newcommand{\hmwkDueDate}{March 15, 2019}
\newcommand{\hmwkClass}{Algorithms and Complexity}
\newcommand{\hmwkClassTime}{}
\newcommand{\hmwkClassInstructor}{Professor Bradford}
\newcommand{\hmwkAuthorName}{\textbf{Super Student} }

%
% Title Page
%

\title{
    \vspace{2in}
    \textmd{\textbf{\hmwkClass:\ \hmwkTitle}}\\
    \normalsize\vspace{0.1in}\small{Due\ on\ \hmwkDueDate\ at 3:10pm}\\
    \vspace{0.1in}\large{\textit{\hmwkClassInstructor\ \hmwkClassTime}}
    \vspace{3in}
}

\author{\hmwkAuthorName}
\date{}

\renewcommand{\part}[1]{\textbf{\large Part \Alph{partCounter}}\stepcounter{partCounter}\\}

%
% Various Helper Commands
%

% Useful for algorithms
\newcommand{\alg}[1]{\textsc{\bfseries \footnotesize #1}}

% For derivatives
\newcommand{\deriv}[1]{\frac{\mathrm{d}}{\mathrm{d}x} (#1)}

% For partial derivatives
\newcommand{\pderiv}[2]{\frac{\partial}{\partial #1} (#2)}

% Integral dx
\newcommand{\dx}{\mathrm{d}x}

% Alias for the Solution section header
\newcommand{\solution}{\textbf{\large Solution}}

% Probability commands: Expectation, Variance, Covariance, Bias
\newcommand{\E}{\mathrm{E}}
\newcommand{\Var}{\mathrm{Var}}
\newcommand{\Cov}{\mathrm{Cov}}
\newcommand{\Bias}{\mathrm{Bias}}

\begin{document}

\maketitle


\pagebreak

\begin{homeworkProblem}

Recall our definition of expander graphs.

Given any graph $G = (V, E)$
and take any subset $S \subseteq V$, we have $S$'s {\em neighbors}
\begin {eqnarray*}
N(S) & = & \{ v \in V: (s,v) \in E \mbox{ for some } s \in S \mbox{ and } v \not\in S \}.
\end {eqnarray*}

A graph expander is a constant degree $d$ graph so that for all $S \subseteq V$ such that $|S| \leq \frac{|V|}{2}$,
\begin {eqnarray*}
|N(S)| \geq c|S|
\end {eqnarray*}
with constant $c > 1$.


A {\em family} of graph expanders with expansion $c$
is an (infinite) set of $d$-regular (bipartite) graphs
$\{ G_i \}_{i=1}^{\infty}$ where each $G_i$ is an $(n_i, d, c)$-expander
and $|V[G_{i+1}]| > |V[G_i]|$ for all $i\geq 1$.

%
%
\vspace{0.1in}
%
%

Assuming an undirected bipartite graph $G=({\cal I} \cup {\cal O}, E)$ where the vertiex partition ${\cal I}$ and ${\cal O}$ are so that

\begin{eqnarray*}
|{\cal I}| & = & |{\cal O}|\\
           & = & \frac{n}{2}.
\end{eqnarray*}

Supose the edges $E$ are made of two bijections $\{ \ \pi_1, \, \pi_2 \ \}$ from ${\cal I}$ to ${\cal O}$.
Thus, $|E| = 2n$.

Prove that $G$ cannot be a graph exapander.\\


    \solution \\


    

\end{homeworkProblem}


\pagebreak


\begin{homeworkProblem}


Suppose $A \subseteq {\cal I}$ and $B \subseteq {\cal O}$ and $|A| = a$ and $|B| = b$ where $a < b$ then
the number of onto functions from $A$ onto $B$:
\begin{eqnarray*}
A & \rightarrow & B
\end{eqnarray*}
is
\begin{eqnarray*}
b(b-1) \cdots (b-a+1) \, (n - a)!
\end{eqnarray*}
since $b - (b-a+1) +1 = a$ so all elements of $A$ are mapped to a unique element of $B$.
Also, $b > a$.
Finally, there are $(n-a)!$ additional mappings of the remaining $n-a$ elements of $A$.
That is, these additional $(n-a)!$ mappings are independent of the other mappings.


Given a bipartite graph $G = ({\cal I} \cup {\cal O}, E)$
where a $1$-factor is a bijection from the inputs ${\cal I}$ to the outputs ${\cal O}$.
Also, $1$-factors may be called permutations.
%
%
\vspace{0.1in}
%
%


\noindent
{\bf Question:} What is the probability of randomly, uniformly selecting 3 bad permutations?

%
%
\vspace{0.1in}
%
%


%                                                                                                                                      
%                                                                                                                                      
%                                                                                                                                      
\begin {theorem}[Most $k \geq 3$-tuples of Random 1-Factors Expand]
{\sf
Randomly, uniformly and independently choose $k$ permutations $\pi = (\pi_1, \pi_2, \cdots, \pi_k)$, for $k  \geq 3$,
each from the set of all permutations on $n$ items.
Use these permutations to build a $k$-degree bipartite graph $G = ({\cal I} \cup {\cal O}, E)$,
then $G$ is an expander with high probability.
}
\label{MOSTRAND}
\end {theorem}

\begin {proof}
Let ${\cal I} = {\cal O} = \{ 1, \cdots, n \}$ and take the $k$-tuple, of
permutations $\pi = (\pi_1, \pi_2, \cdots, \pi_k)$, where $k \geq 3$ and each $\pi_i$ is chosen uniformly and independently at random.

Each $\pi_i$ defines a 1-factor between ${\cal I}$ and ${\cal O}$.\\

How many different choices are there for any such $k$-tuple~$\pi$ ?\\



Now, you must show that it is very unlikely to choose a $k$-tuple of permutations
$\pi$ such that there is a subset $A \subseteq {\cal I}$ where $|A| = n/2$
and at the same time $\pi_i(A) \subseteq B$, for all $i: k \geq i \geq 1$, where $B \subseteq {\cal O}$ and
$|B| \leq (1+c)|A|$



\end{proof}

\end{homeworkProblem}




\pagebreak

\begin{homeworkProblem}

Given the three randonly, uniformly and independently selected permutations from the last problem.

%
%
\vspace{0.1in}
%
%

Is there any way to select them to maximize the expansion coefficient $c$?

\end{homeworkProblem}




%
%
\pagebreak
%
%

%
%
%
\begin{homeworkProblem}


Install the Python libraries from your terminal as shows in Listing~\ref{terminal}.

\begin{lstlisting}[language=bash,label=terminal,caption=Add Python libraries from a omputer terminal]
Computer Prompt> pip3 install numpy networkx matplotlib pygraphviz
\end{lstlisting}


Listing~\ref{libraries} shows Python 3 libraries we use.
These libraries should also be put at the start of Python scripts.\\

\begin{lstlisting}[language=bash,label=libraries,caption=Import libraries in Python]
Python3> import numpy as np
Python3> import networkx as nx
Python3> import matplotlib as mp
Python3> import pygraphviz as pg
\end{lstlisting}

Recall the Erd\H{o}s-R\'enyi random graph model.
The graph in Listing~\ref{ER} is a basic Erd\H{o}s-R\'enyi graph.

\begin{lstlisting}[language=bash,label=ER,caption=Generating basic random graphs in Python]
r5 = nx.generators.random_graphs.fast_gnp_random_graph(5,0.5)

nx.draw(r5, alpha=0.5)
plt.xlim(-1.1, 1.1)
plt.ylim(-1.1, 1.1)
plt.axis('off')
plt.draw()
plt.show()
\end{lstlisting}


The Barab\'asi-Albert (BA) random graph model gives scale-free graphs as discussed in lecture.
Listing~\ref{C+BA} shows how to create a simple BA graph as well as a cycle graph.

\begin{lstlisting}[language=bash,label=C+BA,caption=Two more interesting graphs]
c5 = nx.generators.classic.cycle_graph(5) 
ba = nx.barabasi_albert_graph(50, 2)
\end{lstlisting}

Graph eigenvalues and eigenvectors can be computed using two Python libraries: numpy and networkx.
See Listing~\ref{Eigen}.

\begin{lstlisting}[language=bash,label=Eigen,caption=Eigenvalues and vectors of a random graph]
r5 = nx.generators.random_graphs.fast_gnp_random_graph(5,0.25)

am=nx.adjacency_matrix(r5)
m=np.matrix(am.todense())
ev=np.linalg.eigh(m)
print(ev)
\end{lstlisting}


This exercuse is for you to wirite code and compate the eigenvalues $\lambda_1$ and $\lambda_2$ for both
the Erd\H{o}s-R\'enyi random graphs as well as Barab\'asi-Albert random graphs.




\end{homeworkProblem}


%
%
\pagebreak
%
%

\begin{homeworkProblem}
  If the eigenvalues of the adjacency matrix of a graph are such that $\lambda_1 = \lambda_2$, then the graph is disconnected.
  



\end{homeworkProblem}




\end{document}
