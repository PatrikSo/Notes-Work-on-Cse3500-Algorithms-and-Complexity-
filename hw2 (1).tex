\documentclass{article}

\usepackage{fancyhdr}
\usepackage{extramarks}
\usepackage{amsmath}
\usepackage{amsthm}
\usepackage{amsfonts}
\usepackage{tikz}
\usepackage[plain]{algorithm}
\usepackage{algpseudocode}

\usetikzlibrary{automata,positioning}


\usepackage{listings}
\usepackage{color}
\usepackage{textcomp}

\definecolor{pblue}{rgb}{0.13,0.13,1}
\definecolor{pgreen}{rgb}{0,0.5,0}
\definecolor{pred}{rgb}{0.9,0,0}
\definecolor{pgrey}{rgb}{0.46,0.45,0.48}

\lstset{language=Java,
  showspaces=false,
  showtabs=false,
  breaklines=true,
  showstringspaces=false,
  breakatwhitespace=true,
  commentstyle=\color{pgreen},
  keywordstyle=\color{pblue},
  stringstyle=\color{pred},
  basicstyle=\ttfamily,
  moredelim=[il][\textcolor{pgrey}]{$$},
  moredelim=[is][\textcolor{pgrey}]{\%\%}{\%\%}
}


%
% Basic Document Settings
%

\topmargin=-0.45in
\evensidemargin=0in
\oddsidemargin=0in
\textwidth=6.5in
\textheight=9.0in
\headsep=0.25in

\linespread{1.1}

\pagestyle{fancy}
\lhead{\hmwkAuthorName}
\chead{\hmwkClass\ (\hmwkClassInstructor\ \hmwkClassTime): \hmwkTitle}
\rhead{\firstxmark}
\lfoot{\lastxmark}
\cfoot{\thepage}

\renewcommand\headrulewidth{0.4pt}
\renewcommand\footrulewidth{0.4pt}

\setlength\parindent{0pt}

%
% Create Problem Sections
%

\newcommand{\enterProblemHeader}[1]{
    \nobreak\extramarks{}{Problem \arabic{#1} continued on next page\ldots}\nobreak{}
    \nobreak\extramarks{Problem \arabic{#1} (continued)}{Problem \arabic{#1} continued on next page\ldots}\nobreak{}
}

\newcommand{\exitProblemHeader}[1]{
    \nobreak\extramarks{Problem \arabic{#1} (continued)}{Problem \arabic{#1} continued on next page\ldots}\nobreak{}
    \stepcounter{#1}
    \nobreak\extramarks{Problem \arabic{#1}}{}\nobreak{}
}

\setcounter{secnumdepth}{0}
\newcounter{partCounter}
\newcounter{homeworkProblemCounter}
\setcounter{homeworkProblemCounter}{1}
\nobreak\extramarks{Problem \arabic{homeworkProblemCounter}}{}\nobreak{}

%
% Homework Problem Environment
%
% This environment takes an optional argument. When given, it will adjust the
% problem counter. This is useful for when the problems given for your
% assignment aren't sequential. See the last 3 problems of this template for an
% example.
%
\newenvironment{homeworkProblem}[1][-1]{
    \ifnum#1>0
        \setcounter{homeworkProblemCounter}{#1}
    \fi
    \section{Problem \arabic{homeworkProblemCounter}}
    \setcounter{partCounter}{1}
    \enterProblemHeader{homeworkProblemCounter}
}{
    \exitProblemHeader{homeworkProblemCounter}
}

%
% Homework Details
%   - Title
%   - Due date
%   - Class
%   - Section/Time
%   - Instructor
%   - Author
%

\newcommand{\hmwkTitle}{Homework\ \#2}
\newcommand{\hmwkDueDate}{February 8, 2019}
\newcommand{\hmwkClass}{Algorithms and Complexity}
\newcommand{\hmwkClassTime}{}
\newcommand{\hmwkClassInstructor}{Professor Bradford}
\newcommand{\hmwkAuthorName}{\textbf{Patrik Sokolowski} }

%
% Title Page
%

\title{
    \vspace{2in}
    \textmd{\textbf{\hmwkClass:\ \hmwkTitle}}\\
    \normalsize\vspace{0.1in}\small{Due\ on\ \hmwkDueDate\ at 3:10pm}\\
    \vspace{0.1in}\large{\textit{\hmwkClassInstructor\ \hmwkClassTime}}
    \vspace{3in}
}

\author{\hmwkAuthorName}
\date{}

\renewcommand{\part}[1]{\textbf{\large Part \Alph{partCounter}}\stepcounter{partCounter}\\}

%
% Various Helper Commands
%

% Useful for algorithms
\newcommand{\alg}[1]{\textsc{\bfseries \footnotesize #1}}

% For derivatives
\newcommand{\deriv}[1]{\frac{\mathrm{d}}{\mathrm{d}x} (#1)}

% For partial derivatives
\newcommand{\pderiv}[2]{\frac{\partial}{\partial #1} (#2)}

% Integral dx
\newcommand{\dx}{\mathrm{d}x}

% Alias for the Solution section header
\newcommand{\solution}{\textbf{\large Solution}}

% Probability commands: Expectation, Variance, Covariance, Bias
\newcommand{\E}{\mathrm{E}}
\newcommand{\Var}{\mathrm{Var}}
\newcommand{\Cov}{\mathrm{Cov}}
\newcommand{\Bias}{\mathrm{Bias}}

\begin{document}

\maketitle

\pagebreak

\begin{homeworkProblem}
  Page 67, Exercise 1.




 Suppose you have algorithms with the five running times listed below. (Assume these are the exact running times). How much slower do each of these algorithms get when you (a) Double the input, or (b) Increase the input size by 1?
\vspace{5mm}
\\

n^2
\\
(a)\,\, DOUBLING\,\, INPUT\,\, SIZE
\\
given\,\, running\,\, time = n^2
\\
doubling\,\, input = (2*n)^2 & = & 4*n^2 
\\
gets\,\, 4\,\, times\,\, slower
\\
\vspace{5mm}
\\
(b)\,\, INCREASE\,\, INPUT\,\, SIZE\,\, BY\,\, 1
\\
adding \,\, 1\,\, to\,\, input & = & (n+1)^2 & = & (n+1)(n+1) & = & n^2 + 2n + 1
\\
gets\,\, slower\,\, by\,\, an\,\, additional\,\, 2n+1
\\
-----------------------------------------------------
\\


n^3
\\
(a) \,\, DOUBLING\,\, INPUT\,\, SIZE
\\
given\,\, running\,\, time = n^3
\\
doubling\,\, input = (2*n)^3 & = & 8*n^3 
\\
gets\,\, 8\,\, times\,\, slower
\\
\vspace{5mm}
\\
(b)\,\, INCREASE\,\, INPUT\,\, SIZE\,\, BY\,\, 1
\\
adding \,\, 1\,\, to\,\, input & = & (n+1)^3 & = & (n^2 + 2n + 1)(n+1) & = & n^3 + 3n^2 +3n + 1
\\
gets\,\, slower\,\, by\,\, an\,\, additional\,\, 3n^2 + 3n + 1
\\
-----------------------------------------------------
\\

100n^2
\\
(a)\,\, DOUBLING\,\, INPUT\,\, SIZE     
\\
given\,\, running\,\, time = 100n^2
\\
doubling\,\, input\,\, = 100(2*n)^2 & = & 4 * 100n^2 
\\
gets\,\, 4\,\, times\,\, slower
\\
\vspace{5mm}
\\
(b)\,\, INCREASE\,\, INPUT\,\, SIZE\,\, BY\,\, 1
\\
adding \,\, 1\,\, to\,\, input & = & 100(n+1)^2 & = & 100(n^2 + 2n + 1) & = & 100n^2 + 200n + 100
\\
gets\,\, slower\,\, by\,\, an\,\, additional\,\, 200n + 100
\\
-----------------------------------------------------
\\

nlogn
\\
(a)\,\, DOUBLING\,\, INPUT\,\, SIZE     
\\
given\,\, running\,\, time = 100n^2
\\
doubling\,\, input\,\, = (2*n)log(2*n)
\\
gets\,\, 2\,\, times\,\, slower
\\
\vspace{5mm}
\\
(b) \,\, INCREASE\,\, INPUT\,\, SIZE\,\, BY\,\, 1
\\
doubling\,\, input = (n+1)log(n+1) & = & log(n+1)^{n+1} & = & log(n+1)^n * (n+1) & = & log(n+1)^n * (n+1) * $\frac{n^n}{n^n} $ & = & after some more work & = & nlogn + log(n+1) + n[log(n+1) - logn]
\\
gets\,\, slower\,\, by\,\, an\,\, additional\,\, log(n+1) + n[log(n+1) - logn]
\\
-----------------------------------------------------
\\

2^n
\\
(a)\,\, DOUBLING\,\, INPUT\,\, SIZE
\\
given \,\,running\,\, time = 2^n
\\
doubling\,\, input = 2^{2*n} & = & (2n)^2
\\
gets\,\, 2 \,\,times\,\, slower,\,\, (doubles)
\\    
\vspace{5mm}
\\
(b)\,\, INCREASE\,\, INPUT\,\, SIZE\,\, BY\,\, 1
\\
adding \,\, 1\,\, to\,\, input & = & 2^{n+1} & = & (2^n) * 2
\\
gets\,\, slower\,\, by\,\, an\,\, additional\,\, itself \,\, (doubles)
\\
-----------------------------------------------------
\\


\end{homeworkProblem}



%
%
\pagebreak
%
%

%
%
%
\begin{homeworkProblem}
  Page 67, Exercise 2.

Your\,\, computer\,\, can\,\, perform\,\, 10^{10} \,\,operations\,\, per\,\, second. \,\,For\,\, each\,\, of\,\, the\,\, algorithm's\,\, what\,\, is\,\, the\,\, largest\,\, input\,\, size\,\, n\,\, for\,\, which\,\, you\,\, would\,\, be\,\, able\,\, to\,\, get\,\, the\,\, result\,\, within\,\, an\,\, hour? 
\\    
\vspace{5mm}
\\
Total\,\, number\,\, of\,\, operations\,\, in\,\, an\,\, hour:\,\, 10^{10} \,\,operations/sec\,\, *\,\, 60sec/min\,\, * \,\,60min/hour & = & 3.6*10^{13}
\\
\vspace{2mm}
\\
a) n^2 \,\, largest\,\, input\,\, size\,\, n^2 & = & 3.6*10^{13} \,\,, \,\, n &= & 6,000,000 \,\, operations \,\, ($\sqrt{3.5*10^{13} }) 
\\
b) n^3 \,\, largest\,\, input\,\, size\,\, n^3 & = & 3.6*10^{13} \,\,, \,\, n &= & 33,019 \,\, operations \,\, ($\sqrt[3]{3.5*10^{13} }) 
\\
c) 100n^2 \,\, largest\,\, input\,\, size\,\, 100n^2 & = & 3.6*10^{13} \,\,, \,\, n &= & 600,000 \,\, operations \,\, 
\\
d) nlogn \,\, largest\,\, input\,\, size\,\, nlogn & = & 3.6*10^{13} \,\,, \,\, n &= & 1.29*10^{12} \,\, operations \,\,
\\
e) 2^n \,\, largest\,\, input\,\, size\,\, 2^n & = & 3.6*10^{13} \,\,, \,\, n &= & 45 \,\, operations \,\, 
\\
f) 2^{2^n} \,\, largest\,\, input\,\, size\,\, 2^{2^n} & = & 3.6*10^{13} \,\,, \,\, n &= & 5 \,\, operations \,\, 
\\









\end{homeworkProblem}


%
%
\pagebreak
%
%

\begin{homeworkProblem}
  Page 67, Exercise 3.
\vspace{5mm}
\\
Arrange\,\, these\,\, algorithms\,\, in\,\, ascending\,\, order\,\, of\,\, growth\,\, rate.
\\
\vspace{5mm}
\\
f_1 (n) &=& n^{2.5}  \,\,\,\,\,\,\,\,\,\,\,\,\,\, exponential,\,\, higher\,\, than\,\, f_2 \,\, and\,\, f_3
\\
f_2 (n) &=& $\sqrt{2n}  \,\,\,\,\,\,\,\,\,\,\,\,\,\, lowest\,\, degree
\\
f_3 (n) &=& n+10   \,\,\,\,\,\,\,\,\,\,\,\,\,\, higher\,\, degree\,\, of\,\, 1\,\, than\,\, f_2
\\
f_4 (n) &=& 10^n   \,\,\,\,\,\,\,\,\,\,\,\,\,\, highest\,\, degree/most\,\, exponential
\\
f_5 (n) &=& 100^n   \,\,\,\,\,\,\,\,\,\,\,\,\,\, highest\,\, degree/most\,\, exponential
\\
f_6 (n) &=& n^2logn   \,\,\,\,\,\,\,\,\,\,\,\,\,\, exponential,\,\, higher\,\, than\,\, f_2 \,\,and\,\,
 f_3
\\
\vspace{5mm}
\\
f_2 \,\, \textless \,\, f_3 \,\, \textless \,\, f_6 \,\, \textless \,\, f_1 \,\, \textless \,\, f_4 \,\, \textless \,\, f_5


\end{homeworkProblem}


%
%
\pagebreak
%
%

\begin{homeworkProblem}
  Page 68, Exercise 6.

Given an Array A consisting of n intergers A[1], A[2], . . . , A[n]. You'd like to output a 2 dimensional n by n array B in which B[i,j](for i<j) continues the sum of array entries A[i] through A[j].
\\
\vspace{5mm}
\\
Given Algorithm:\\
.\,\,\,\,\,\,\,\, For i = 1, 2, ... n \\
.\,\,\,\,\,\,\,\, \,\,\,\,\,\,\,\, For j = i+1, i+2, ..., n \\
.\,\,\,\,\,\,\,\, \,\,\,\,\,\,\,\, \,\,\,\,\,\,\,\, Add up array entries A[i] through A[j]\\
.\,\,\,\,\,\,\,\, \,\,\,\,\,\,\,\, \,\,\,\,\,\,\,\, Store the result in B[i,j] \\
.\,\,\,\,\,\,\,\, \,\,\,\,\,\,\,\, End for \\
.\,\,\,\,\,\,\,\, End for \\
\\

A) \,\, For some function f that you should choose, give a bound of the form O(f(n)) on the running time of this algorithm on an input of size n (i,e, a bound on the number of operations performed by the algorithm).
\\
\vspace{2mm}
\\
O(n^3 )
\\
Outer for-loop\,\, goes\,\, on\,\, n\,\, times\,\, O(n)
\\
Inner\,\, for-loop\,\, goes\,\, n \,\,times\,\, for\,\, the\,\, outer\,\, loop\,\, (another O(n) )
\\
adding\,\, up\,\, the\,\, array\,\, entries\,\, is\,\, O(n)\,\, times,\,\, inside\,\, the\,\, loops
\\
resulting\,\, in\,\, O(n^3 )
\\

B) \,\, For the same function f, show that the running time of the algorithm an on input size n is also asymptotically tight bound of O(f(n)) on the running time.
\\
\vspace{2mm}
\\
Outer loop goes through n times
\\
Iterations od inner loop: (n-1) + (n+1) + ... 1 + 0 &=& (1/2)(n-1)((n+1)-1) = (1/2)n^2 - (1/2)n
\\
Number\,\, of\,\, operations\,\, on\,\, adding\,\, for\,\, n\,\, iterations\,\, of\,\, inner\,\, loop:
\\
For i = 1: 1+2+...+n-1 = (1/2)(n-1)(n) = 0.5n^2 - (1/2)n
\\
For i = 2: 1+2+...+n-2 = (1/2)(n-2)(n-1) = 0.5n^2 - (1/2)n
\\
For i = k: 1+2+...+n-k = (1/2)(n-k)(n(k+1)) 
\\
so for i = n-1 : 1+(n-(n-1)) = 1+2 = n^2 /8
\\
There\,\, are\,\, at\,\, least\,\, n^3 /16\,\, addition\,\, operations\,\, for\,\, all\,\, n \textmore = 2
\\
Algorithm\,\, is\,\, lower\,\, bounded\,\, by\,\, n^3
\\

C) \,\, Give a different algorithm to solve the problem, with an assymptotically better running time.
\\
Algorithm with (O(n^2)) instead\,\, of\,\, O(n^3 )):
\\
indexedSum = 0;
\\
.\,\,\,\,\,\,\,\, For i = 1, 2, ... n
\\
.\,\,\,\,\,\,\,\,\,\,\,\,\,\,\,\, indexedSum = A[i];
\\
.\,\,\,\,\,\,\,\,\,\,\,\,\,\,\,\, For j = i+1, i+2, ... n
\\
.\,\,\,\,\,\,\,\,\,\,\,\,\,\,\,\,\,\,\,\,\,\,\,\, indexedSum += A[j]
\\
.\,\,\,\,\,\,\,\,\,\,\,\,\,\,\,\,\,\,\,\,\,\,\,\, Store\,\, indexedSum \,\, in\,\, B[i,j]
\\
.\,\,\,\,\,\,\,\,\,\,\,\,\,\,\,\,End\,\, for
\\
.\,\,\,\,\,\,\,\,End\,\, for

\end{homeworkProblem}


\end{document}
