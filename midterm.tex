\documentclass{article}

\usepackage{fancyhdr}
\usepackage{extramarks}
\usepackage{amsmath}
\usepackage{amsthm}
\usepackage{amsfonts}
\usepackage{tikz}
\usepackage[plain]{algorithm}
\usepackage{algpseudocode}

\usetikzlibrary{automata,positioning}


\usepackage{listings}
\usepackage{color}
\usepackage{textcomp}

\newtheorem{theorem}{Theorem}
\newtheorem{property}{Property}
\newtheorem{lemma}{Lemma}
\newtheorem{fact}{Fact}
\newtheorem{definition}{Definition}
\newtheorem{corollary}{Corollary}
\newtheorem{convention}{Convention}
\newtheorem{remark}{Remark}
\newtheorem{observation}{Observation}
\newtheorem{conjecture}{Conjecture}


\definecolor{pblue}{rgb}{0.13,0.13,1}
\definecolor{pgreen}{rgb}{0,0.5,0}
\definecolor{pred}{rgb}{0.9,0,0}
\definecolor{pgrey}{rgb}{0.46,0.45,0.48}




\lstdefinelanguage{bash}{
  keywords={sudo, apt-get, ifconfig, mosquitto_pub, mosquitto_sub, ping},
  basicstyle=\small,
  alsoletter=?!-,
  alsodigit=\$\%&*+./:<=>@^_~,
  sensitive=false,
  morecomment=[l]{\#},
  morecomment=[s]{/*}{*/},
  morestring=[b]'',
  basicstyle=\small\ttfamily,
  keywordstyle=\bf\ttfamily\color[rgb]{0,.3,.7},
  commentstyle=\color[rgb]{0.133,0.545,0.133},                                                                            
  stringstyle={\color[rgb]{0.75,0.49,0.07}},
  upquote=true,
  breaklines=true,
  breakatwhitespace=true,
  literate=*{`}{{`}}{1},
  keywords=[2]{myRed},
  keywordstyle=[2]\color{red}
}

\lstset{language=Python,
  keywords={nx,plt,np,mp,pg},
  showspaces=false,
  showtabs=false,
  breaklines=true,
  showstringspaces=false,
  breakatwhitespace=true,
  commentstyle=\color{pgreen},
  keywordstyle=\color{red},
  stringstyle=\color{pred},
  basicstyle=\ttfamily,
  moredelim=[il][\textcolor{pgrey}]{$$},
  moredelim=[is][\textcolor{pgrey}]{\%\%}{\%\%}
}


%
% Basic Document Settings
%

\topmargin=-0.45in
\evensidemargin=0in
\oddsidemargin=0in
\textwidth=6.5in
\textheight=9.0in
\headsep=0.25in

\linespread{1.1}

\pagestyle{fancy}
\lhead{\hmwkAuthorName}
\chead{\hmwkClass\ (\hmwkClassInstructor\ \hmwkClassTime): \hmwkTitle}
\rhead{\firstxmark}
\lfoot{\lastxmark}
\cfoot{\thepage}

\renewcommand\headrulewidth{0.4pt}
\renewcommand\footrulewidth{0.4pt}

\setlength\parindent{0pt}

%
% Create Problem Sections
%

\newcommand{\enterProblemHeader}[1]{
    \nobreak\extramarks{}{Problem \arabic{#1} continued on next page\ldots}\nobreak{}
    \nobreak\extramarks{Problem \arabic{#1} (continued)}{Problem \arabic{#1} continued on next page\ldots}\nobreak{}
}

\newcommand{\exitProblemHeader}[1]{
    \nobreak\extramarks{Problem \arabic{#1} (continued)}{Problem \arabic{#1} continued on next page\ldots}\nobreak{}
    \stepcounter{#1}
    \nobreak\extramarks{Problem \arabic{#1}}{}\nobreak{}
}

\setcounter{secnumdepth}{0}
\newcounter{partCounter}
\newcounter{midtermProblemCounter}
\setcounter{midtermProblemCounter}{1}
\newcounter{homeworkProblemCounter}
\setcounter{homeworkProblemCounter}{1}
\nobreak\extramarks{Problem \arabic{homeworkProblemCounter}}{}\nobreak{}

%
% Homework Problem Environment
%
% This environment takes an optional argument. When given, it will adjust the
% problem counter. This is useful for when the problems given for your
% assignment aren't sequential. See the last 3 problems of this template for an
% example.
%

\newenvironment{midtermProblem}[1][-1]{
    \ifnum#1>0
        \setcounter{midtermProblem}{#1}
    \fi
    \section{Problem \arabic{midtermProblemCounter}}
    \setcounter{partCounter}{1}
    \enterProblemHeader{midtermProblemCounter}
}{
    \exitProblemHeader{midtermProblemCounter}
} 

\newenvironment{homeworkProblem}[1][-1]{
    \ifnum#1>0
        \setcounter{homeworkProblemCounter}{#1}
    \fi
    \section{Problem \arabic{homeworkProblemCounter}}
    \setcounter{partCounter}{1}
    \enterProblemHeader{homeworkProblemCounter}
}{
    \exitProblemHeader{homeworkProblemCounter}
}

%
% Homework Details
%   - Title
%   - Due date
%   - Class
%   - Section/Time
%   - Instructor
%   - Author
%

\newcommand{\hmwkTitle}{Midterm}
\newcommand{\hmwkDueDate}{April 5, 2019}
\newcommand{\hmwkClass}{Algorithms and Complexity}
\newcommand{\hmwkClassTime}{}
\newcommand{\hmwkClassInstructor}{Professor Bradford}
\newcommand{\hmwkAuthorName}{\textbf{Patrik Sokolowski} }

%
% Title Page
%

\title{
    \vspace{2in}
    \textmd{\textbf{\hmwkClass:\ \hmwkTitle}}\\
    \normalsize\vspace{0.1in}\small{Due\ on\ \hmwkDueDate\ at the start of class}\\
    \vspace{0.1in}\large{\textit{\hmwkClassInstructor\ \hmwkClassTime}}
    \vspace{3in}
}

\author{\hmwkAuthorName}
\date{}

\renewcommand{\part}[1]{\textbf{\large Part \Alph{partCounter}}\stepcounter{partCounter}\\}

%
% Various Helper Commands
%

% Useful for algorithms
\newcommand{\alg}[1]{\textsc{\bfseries \footnotesize #1}}

% For derivatives
\newcommand{\deriv}[1]{\frac{\mathrm{d}}{\mathrm{d}x} (#1)}

% For partial derivatives
\newcommand{\pderiv}[2]{\frac{\partial}{\partial #1} (#2)}

% Integral dx
\newcommand{\dx}{\mathrm{d}x}

% Alias for the Solution section header
\newcommand{\solution}{\textbf{\large Solution}}

% Probability commands: Expectation, Variance, Covariance, Bias
\newcommand{\E}{\mathrm{E}}
\newcommand{\Var}{\mathrm{Var}}
\newcommand{\Cov}{\mathrm{Cov}}
\newcommand{\Bias}{\mathrm{Bias}}

\newcommand{\id}{\mbox{\bf \em id}}
\newcommand{\lo}{\mbox{\bf \em loc}}
\newcommand{\lcm}{\mbox{\rm lcm}}
\newcommand{\g}{\mbox{\bf \em g}}
\newcommand{\reg}{\mbox{\bf \em r}}
\newcommand{\B}{\mbox{\bf Ball}}

\begin{document}

\maketitle


\pagebreak

\begin{midtermProblem}

Recall our definition of expander graphs.


%
%
\vspace{0.1in}
%
%

Assuming an undirected bipartite graph $G=({\cal I} \cup {\cal O}, E)$ where the vertex partition ${\cal I}$ and ${\cal O}$ are so that

\begin{eqnarray*}
|{\cal I}| & = & |{\cal O}| \ \ = \ \ p^2\\
\end{eqnarray*}
for some prime $p > 2$.

Now, take all inputs ${\cal I}$ as ${\cal I} = \displaystyle \bigcup_{j=0}^{p-1} I_j$ where each $I_j$ is an input block so $|I_j| = p$.
Likewise, for all outputs ${\cal O}$ it must be that ${\cal O} = \displaystyle \bigcup_{j=0}^{p-1} O_j$ where each $O_j$ is an output block
where $|O_j| = p$.

Define $\mathbb{Z}_{p}^{+} = \{ 0, 1, \cdots, p-1 \}$ where addition is computed mod $p$.

For any input node $(j,k) \in I_j$ is position $k$ in block $j$ where
$j \in \mathbb{Z}_{p}^{+}$ and $k \in \mathbb{Z}_{p}^{+}$.
The graph $G_4$ has the following sets of edges:\\
\begin{enumerate}

\item {\em Identity}:
$\id(j,k) \ \longrightarrow \ (j,k)$.

\item {\em Local Shift}:
$\lo (j,k) \ \longrightarrow \ (j, \ (j+k+1) \bmod p)$.

\item {\em Global Shift}:
$\g (j,k) \ \longrightarrow \ ((j+k) \bmod p, \ k)$.
\end{enumerate}
\noindent
These edges are directed from the inputs ${\cal I}$ to the outputs ${\cal O}$. 
The adversary can only select inputs to prevent expansion among the outputs.
In general, for block $I_j$, an adversary selects input elements $\widehat{I}_j \subseteq I_j$
and these selected inputs have output neighbors $\widehat{O} \subseteq {\cal O}$.



The global shift $\g$ has the inverse $\g^{-1}(j,k) = (j-k,k)$ since $\g(j+k-k,k) = (j,k)$ computed mod-$p$.
Further, $\g_1(j,k), \lo_1(j,k), \id_1(j,k),$ denotes the first 
component of the pair, while $\g_2(j,k), \lo_2(j,k), \id_2(j,k)$.


\begin{figure}[ht]
 %\vspace{-0.5cm}
 \begin{center}
 \includegraphics[height=8cm,width=5cm]{5x5-global.png}
 \end{center}
 \vspace{-0.6cm}
\caption{Selected edges of global permutation for $p=5$}
\label{fig:global}
\end{figure}


\begin{figure}[ht]
 %\vspace{-0.5cm}
 \begin{center}
 \includegraphics[height=8cm,width=5cm]{5x5-local.png}
 \end{center}
 \vspace{-0.6cm}
\caption{Local edges from $\lo$ for $p=5$}
\label{fig:global}
\end{figure}

\vspace{0.2in}

\fbox{Our focus is on showing expansion with exactly $p(p-1)/2$ inputs.}

\vspace{0.2in}

{\bf Question 1}:  Show that given the local and identity permutations `most' of the inputs must follow each local pattern.\\
\vspace{20mm}\\
{\bf Question 2}:  Show that given the global permutations `most' of the inputs must follow the global pattern.\\
\vspace{20
mm}\\
{\bf Question 3}:  How may we combine the local, id and global permutations to show we have expansion?\\

    

\end{midtermProblem}


\pagebreak


\begin{midtermProblem}

P 246, \# 4.

  \begin{center}
  \includegraphics[scale = .70]{problem3figure1.PNG}\\
  \includegraphics[scale = .70]{problem3figure2.PNG}\\
  \end{center}

We can use convolution.\\ 
If we have a vector set of (q_1 , q_2 , ... , q_n )$  and a vector set of (n^{-2} , (n-1)^{-2} , ... , 1/4 , 0 , -1 , -1/4 , ... , -n^{-2} )$\\
For each j we can put the convolution of both vector sets and have the equivalent of the coulumb equation:\\

\sum_{i<j} \frac{q_i}{ (j-i)^2 } + \sum_{i<j} \frac{-q_i}{ (j-i)^2 }\\

We simply multiply every $q_i with Cq_i $in the end to get the full F_j.\\
$This will get the faster running time they wanted.\\

\end{midtermProblem}









\end{document}
