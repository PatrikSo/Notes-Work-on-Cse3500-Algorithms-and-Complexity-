\documentclass{article}

\usepackage{fancyhdr}
\usepackage{extramarks}
\usepackage{amsmath}
\usepackage{amsthm}
\usepackage{amsfonts}
\usepackage{tikz}
\usepackage[plain]{algorithm}
\usepackage{algpseudocode}

\usetikzlibrary{automata,positioning}


\usepackage{listings}
\usepackage{color}
\usepackage{textcomp}

\definecolor{pblue}{rgb}{0.13,0.13,1}
\definecolor{pgreen}{rgb}{0,0.5,0}
\definecolor{pred}{rgb}{0.9,0,0}
\definecolor{pgrey}{rgb}{0.46,0.45,0.48}

\lstset{language=Java,
  showspaces=false,
  showtabs=false,
  breaklines=true,
  showstringspaces=false,
  breakatwhitespace=true,
  commentstyle=\color{pgreen},
  keywordstyle=\color{pblue},
  stringstyle=\color{pred},
  basicstyle=\ttfamily,
  moredelim=[il][\textcolor{pgrey}]{$$},
  moredelim=[is][\textcolor{pgrey}]{\%\%}{\%\%}
}


%
% Basic Document Settings
%

\topmargin=-0.45in
\evensidemargin=0in
\oddsidemargin=0in
\textwidth=6.5in
\textheight=9.0in
\headsep=0.25in

\linespread{1.1}

\pagestyle{fancy}
\lhead{\hmwkAuthorName}
\chead{\hmwkClass\ (\hmwkClassInstructor\ \hmwkClassTime): \hmwkTitle}
\rhead{\firstxmark}
\lfoot{\lastxmark}
\cfoot{\thepage}

\renewcommand\headrulewidth{0.4pt}
\renewcommand\footrulewidth{0.4pt}

\setlength\parindent{0pt}

%
% Create Problem Sections
%

\newcommand{\enterProblemHeader}[1]{
    \nobreak\extramarks{}{Problem \arabic{#1} continued on next page\ldots}\nobreak{}
    \nobreak\extramarks{Problem \arabic{#1} (continued)}{Problem \arabic{#1} continued on next page\ldots}\nobreak{}
}

\newcommand{\exitProblemHeader}[1]{
    \nobreak\extramarks{Problem \arabic{#1} (continued)}{Problem \arabic{#1} continued on next page\ldots}\nobreak{}
    \stepcounter{#1}
    \nobreak\extramarks{Problem \arabic{#1}}{}\nobreak{}
}

\setcounter{secnumdepth}{0}
\newcounter{partCounter}
\newcounter{homeworkProblemCounter}
\setcounter{homeworkProblemCounter}{1}
\nobreak\extramarks{Problem \arabic{homeworkProblemCounter}}{}\nobreak{}

%
% Homework Problem Environment
%
% This environment takes an optional argument. When given, it will adjust the
% problem counter. This is useful for when the problems given for your
% assignment aren't sequential. See the last 3 problems of this template for an
% example.
%
\newenvironment{homeworkProblem}[1][-1]{
    \ifnum#1>0
        \setcounter{homeworkProblemCounter}{#1}
    \fi
    \section{Problem \arabic{homeworkProblemCounter}}
    \setcounter{partCounter}{1}
    \enterProblemHeader{homeworkProblemCounter}
}{
    \exitProblemHeader{homeworkProblemCounter}
}

%
% Homework Details
%   - Title
%   - Due date
%   - Class
%   - Section/Time
%   - Instructor
%   - Author
%

\newcommand{\hmwkTitle}{Homework\ \#1}
\newcommand{\hmwkDueDate}{February 1, 2019}
\newcommand{\hmwkClass}{Algorithms and Complexity}
\newcommand{\hmwkClassTime}{}
\newcommand{\hmwkClassInstructor}{Professor Bradford}
\newcommand{\hmwkAuthorName}{\textbf{Patrik Sokolowski} }

%
% Title Page
%

\title{
    \vspace{2in}
    \textmd{\textbf{\hmwkClass:\ \hmwkTitle}}\\
    \normalsize\vspace{0.1in}\small{Due\ on\ \hmwkDueDate\ at 3:10pm}\\
    \vspace{0.1in}\large{\textit{\hmwkClassInstructor\ \hmwkClassTime}}
    \vspace{3in}
}

\author{\hmwkAuthorName}
\date{}

\renewcommand{\part}[1]{\textbf{\large Part \Alph{partCounter}}\stepcounter{partCounter}\\}

%
% Various Helper Commands
%

% Useful for algorithms
\newcommand{\alg}[1]{\textsc{\bfseries \footnotesize #1}}

% For derivatives
\newcommand{\deriv}[1]{\frac{\mathrm{d}}{\mathrm{d}x} (#1)}

% For partial derivatives
\newcommand{\pderiv}[2]{\frac{\partial}{\partial #1} (#2)}

% Integral dx
\newcommand{\dx}{\mathrm{d}x}

% Alias for the Solution section header
\newcommand{\solution}{\textbf{\large Solution}}

% Probability commands: Expectation, Variance, Covariance, Bias
\newcommand{\E}{\mathrm{E}}
\newcommand{\Var}{\mathrm{Var}}
\newcommand{\Cov}{\mathrm{Cov}}
\newcommand{\Bias}{\mathrm{Bias}}

\begin{document}

\maketitle

\pagebreak

\begin{homeworkProblem}
  Page 22, Exercise 1.

  


    \solution \\
    \textbf{Explanation or Counterexample}\\

  \(True or False?\) In every instance of the Stable Matching Problem, There is
  a stable matching containing a pair (m, w) such that m is ranked first on the
  preference list of w and w is ranked first on the preference list of m.
  
  \[
  False!

  Consider:
  \]
  \begin{center}
  \includegraphics{Male Female preference lists.PNG}
  Picture Provided by Professor Bradford's Power Point 
  \end{center}

  m1 prefers w1 and m2 prefers w1
  w1 prefers m1 and w2 prefers m1
  \newline
  \\
  It is impossible to have all four people have their number 1 preferences if
  two on each side prefer one on the other side.
  \\
  Having m1 and m2 prefer w1 will make it impossible to satisfy this unless a 
  polygamous rule is put in. Both m's cannot be made into a perfect match.
  \\
  Having w1 and w2 prefer m1 wil make it impossible to satisfy this unless a 
  polygamous rule is put in. Both w's cannot be made into a perfect match.

\end{homeworkProblem}



%
%
\pagebreak
%
%

%
%
%
\begin{homeworkProblem}
  Page 22, Exercise 2.

    \solution \\
    \textbf{Explanation or Counterexample}\\

  \(True or False?\) Consider an instance of the Stable Matching Problem in which
  there exists a man m and a woman w such that m is ranked first on the preference
  list of w and w is ranked first in the preference list of m. Then, in every stable matching
  S for this intance, the pair (m, w) belongs to S.

  \[
  True!

  Consider the pairs:
 \\
  (m, w^1 ) and ( m^1 , w)
  \]

  We know m prefers w and w prefers m. If given any different match-ups such as the ones
  above we get instabilities which are not perfect pairs. This would not hold in this instance,
  only showing (m, w) belongs in S. We wouldn't be able to satisfy the instance which wants
  (m,w) belonging to S, so we need (m, w) matched.
  

\end{homeworkProblem}


%
%
\pagebreak
%
%

\begin{homeworkProblem}

Give an inductive proof of 

\begin{eqnarray*}
\sum_{i=0}^{n-1} (2i+1) & = & n^2.
\end{eqnarray*}

    \solution \\

    \begin{proof}
      Put your proof here.
      \end{proof}

Proof:

\begin{eqnarray*}
Basis: n = 1
\\
Left side: &
\sum_{i=0}^{1-1} (2(0)+1) & = 1
\\
Right side: &
(1)^2 & = 1
\end{eqnarray*}


\\ 
Both sides are equal, making it true for n=1

\begin{eqnarray*}
Induction Step: assuming \, (n = 1) \, is \, true \, for \, (n = c)
\\
\sum_{i=0}^{c-1+1} (2i+1) & = & \sum_{i=0}^{c-1} (2i+1) + (2(c+1)-1)
\\
= & c^2 + 2(c+1) - 1 & (induction hypothesis)
\\
= & c^2 + 2c + 1 & =  (c+1)^2  (right side)
\\
\end{eqnarray*}
So this must hold for all n\leq
2


\end{homeworkProblem}


%
%
\pagebreak
%
%

\begin{homeworkProblem}
Let 
\begin{eqnarray}
f(n) & = & \sum_{i=0}^{n-1} x^i, \label{MainEQ}
\end{eqnarray}

\noindent
Do each of the following,\\


 \part

 Prove $f(n) = f(n-1) + x^{n-1}$

 %
 %
 \vspace{0.1in}
 %
 %


    \begin{proof}
      \\
     $ Basis: f(2) = f(1)
      \\
      f(1) & = & \sum_{i=0}^{1-1} x^i & = & 1
      \\
      so \,\,\,\,\,& f(2) & = & 1 + x
      \\
      and \,\,\,\,\,& \sum_{i=0}^{n-1} x^i & = & {1 + x} \,\,\,\,\, for \,\,\,\,\, n=2$
      \\
      \\
      \\
      Inductive hypothesis:
      \\
      say for n\leq k, \,\,\, some \,\,\, k
      \\
      f(n) & = & \sum_{i=0}^{k-1} x^i
      \\
      Suppose, FSOC (For\,\, Sake\,\, Of\,\, Contradiction), k \,\, is\,\, the\,\, smallest\,\, integer\,\, where\,\,\,\,  f(k) & \neq & \sum_{i=0}^{k-1} x^i
      \\
      by\,\, definition,\,\, f(k) & = & f(k-1) & + & x^{k-1} 
      \\
      However\,\, f(k-1) = \sum_{i=0}^{k-2} x^i
      \\
      and\,\, f(k) & = & f(k-1) & + & x^{k-1}
      \\
      This\,\, means\,\, there\,\, can\,\, be\,\, no\,\, minimal\,\, k\,\, where
      \\
      f(k) & \neq & \sum_{i=0}^{k-1} x^i
    \end{proof}

 %
 %
 \vspace{1in}
 %
 %


 \part 

 Prove $f(n) = xf(n-1) + 1$

 %
 %
 \vspace{0.1in}
 %
 %


    \begin{proof}
      \\
     $ Basis: n = 2
      \\
      f(2) & = & xf(1) & + & 1
      \\
      \sum_{i=0}^{1-1} x^i & + & 1
      \\
      f(2) = x + 1
      \\
      \\
      Inductive\,\, hypothesis:
      f(n) = xf(n-1) + 1 \,\,\,\, \forall \,\,\,\, n \leq C
      \\
      Inductive\,\, Step:
      f(n)\,\, fails\,\, first\,\, for\,\, n
      \\
      f(n) = xf(n-1)+1,\,\, by \,\, I.H:
      \\
      \sum_{i=0}^{n-2} x^i & + & 1 \,\,\,\,\,\,\,\,\,\, \sum_{i=0}^{n-2} x^{i + 1}
      \\
      f(n) = \sum_{i=1}^{n-1} x^i & + & 1  \,\,\,\,\,\,\,\,\,\,  \sum_{i=1}^{n-1} x^{i + 1}



    \end{proof}
 %
 %
 \vspace{1in}
 %
 %


 \part 

 Prove $f(n) = (x^n-1)/(x-1)$
 
 %
 %
 \vspace{0.1in}
 %
 %
    \begin{proof}
      \\
     $ Basis: f(2) = f(1)
      \\
      f(1) & = & \sum_{i=0}^{1-1} x^i & = & 1
      \\
      so \,\,\,\,\,& f(2) & = & 1 + x
      \\
      and \,\,\,\,\,& \sum_{i=0}^{n-1} x^i & = & {1 + x} \,\,\,\,\, for \,\,\,\,\, n=2$
      \\
      \\
      \\
      Inductive hypothesis:
      \\
      say for n\leq k, \,\,\, some \,\,\, k
      \\
      f(n) & = & \sum_{i=0}^{k-1} x^i
      \\
      Suppose, FSOC (For\,\, Sake\,\, Of\,\, Contradiction), k \,\, is\,\, the\,\, smallest\,\, integer\,\, where\,\,\,\,  f(k) & \neq & \sum_{i=0}^{k-1} x^i
      \\
      by\,\, definition,\,\, f(k) & = & (x^{n} -1) / (x-1) 
      \\
      However\,\, f(2) & = & 1 + x & \neq & (x^{2} -1) / (x-1) 
      \\
      and\,\, f(k) & = & (x^{2} -1) / (x-1) 
      \\
      This\,\, means\,\, there\,\, can\,\, be\,\, no\,\, minimal\,\, k\,\, where
      \\
      f(k) & \neq & \sum_{i=0}^{k-1} x^i      
    \end{proof}

 %
 %
 \vspace{1in}
 %
 %

 \part 

 What happens when $x=1$, how about for Equation~\ref{MainEQ}?
\\
f(n) & = & \sum_{i=0}^{n-1} 1^i
\\
f(1) & = & \sum_{i=0}^{1-1} 1^i & = & 1 \,\,\,\,\,\,\,\,\,\,\,\,\,\,\,\,\,\,\,\,\,=\,\,\,\,\,\,\,1\\
f(2) & = & \sum_{i=0}^{2-1} 1^i & = & 1 + 1 \,\,\,\,\,\,\,\,\,\,\,=\,\,\,\,\,\,\,2\\
f(3) & = & \sum_{i=0}^{3-1} 1^i & = & 1 + 1 + 1 \, = \,\,\,\,\,\,\,3\\
f(n) & = & \sum_{i=0}^{n} n
\\





Part A:
\\
f(n) & = & f(n-1)&x^{n-1}
\\
f(n) & = & f(n-1)&1^{n-1}
\\
Basis: f(1) = 1 &\,\, and\,\, & f(2) = 2
\\
Suppose, FSOC (For\,\, Sake\,\, Of\,\, Contradiction), k \,\, is\,\, the\,\, smallest\,\, integer\,\, where\,\,\,\,  f(k) & \neq & \sum_{i=0}^{k-1} x^i
\\
by\,\, definition,\,\, f(k) & = & f(k-1)&1^{k-1} 
\\
However, f(1) & = & 0 & \neq & \sum_{i=0}^{1-1} 1^i & = & 1
\\
This\,\, means\,\, there\,\, can\,\, be\,\, no\,\, minimal\,\, k\,\, where
\\
 f(k) & \neq & \sum_{i=0}^{k-1} x^i
\\




Part B:
\\
Basis: n = 1
\\
f(n) & = & \sum_{i=0}^{1-1} 1^i & = & (1)f(1-1)+1
\\
Inductive hypothesis: 
\\
\sum_{i=0}^{n-1} 1^i & = & (1)f(n-1)+1 \,\, \forall \,\, n \,\, \leq \,\, C
\\
Inductive Step:
\\
 (1)f(n-1)+1 \,\, always \,\, equals \,\, \sum_{i=0}^{n-1} 1^i \,\, 
\\
f(n-1)+1 & = & f(n) 
\\
by \,\, definition \,\, will = &  \sum_{i=0}^{n-1} 1^i
\\




Part C:
\\
f(n) & = & (x^{2} -1) / (x-1)
\\
f(n) & = & (1^{2} -1) / (1-1)
\\
f(n) = 0
\end{homeworkProblem}

\end{document}
