\documentclass{article}

\usepackage{fancyhdr}
\usepackage{extramarks}
\usepackage{amsmath}
\usepackage{amsthm}
\usepackage{amsfonts}
\usepackage{tikz}
\usepackage[plain]{algorithm}
\usepackage{algpseudocode}

\usetikzlibrary{automata,positioning}


\usepackage{listings}
\usepackage{color}
\usepackage{textcomp}

\definecolor{pblue}{rgb}{0.13,0.13,1}
\definecolor{pgreen}{rgb}{0,0.5,0}
\definecolor{pred}{rgb}{0.9,0,0}
\definecolor{pgrey}{rgb}{0.46,0.45,0.48}

\lstset{language=Java,
  showspaces=false,
  showtabs=false,
  breaklines=true,
  showstringspaces=false,
  breakatwhitespace=true,
  commentstyle=\color{pgreen},
  keywordstyle=\color{pblue},
  stringstyle=\color{pred},
  basicstyle=\ttfamily,
  moredelim=[il][\textcolor{pgrey}]{$$},
  moredelim=[is][\textcolor{pgrey}]{\%\%}{\%\%}
}


%
% Basic Document Settings
%

\topmargin=-0.45in
\evensidemargin=0in
\oddsidemargin=0in
\textwidth=6.5in
\textheight=9.0in
\headsep=0.25in

\linespread{1.1}

\pagestyle{fancy}
\lhead{\hmwkAuthorName}
\chead{\hmwkClass\ (\hmwkClassInstructor\ \hmwkClassTime): \hmwkTitle}
\rhead{\firstxmark}
\lfoot{\lastxmark}
\cfoot{\thepage}

\renewcommand\headrulewidth{0.4pt}
\renewcommand\footrulewidth{0.4pt}

\setlength\parindent{0pt}

%
% Create Problem Sections
%

\newcommand{\enterProblemHeader}[1]{
    \nobreak\extramarks{}{Problem \arabic{#1} continued on next page\ldots}\nobreak{}
    \nobreak\extramarks{Problem \arabic{#1} (continued)}{Problem \arabic{#1} continued on next page\ldots}\nobreak{}
}

\newcommand{\exitProblemHeader}[1]{
    \nobreak\extramarks{Problem \arabic{#1} (continued)}{Problem \arabic{#1} continued on next page\ldots}\nobreak{}
    \stepcounter{#1}
    \nobreak\extramarks{Problem \arabic{#1}}{}\nobreak{}
}

\setcounter{secnumdepth}{0}
\newcounter{partCounter}
\newcounter{homeworkProblemCounter}
\setcounter{homeworkProblemCounter}{1}
\nobreak\extramarks{Problem \arabic{homeworkProblemCounter}}{}\nobreak{}

%
% Homework Problem Environment
%
% This environment takes an optional argument. When given, it will adjust the
% problem counter. This is useful for when the problems given for your
% assignment aren't sequential. See the last 3 problems of this template for an
% example.
%
\newenvironment{homeworkProblem}[1][-1]{
    \ifnum#1>0
        \setcounter{homeworkProblemCounter}{#1}
    \fi
    \section{Problem \arabic{homeworkProblemCounter}}
    \setcounter{partCounter}{1}
    \enterProblemHeader{homeworkProblemCounter}
}{
    \exitProblemHeader{homeworkProblemCounter}
}

%
% Homework Details
%   - Title
%   - Due date
%   - Class
%   - Section/Time
%   - Instructor
%   - Author
%

\newcommand{\hmwkTitle}{Homework\ \#3}
\newcommand{\hmwkDueDate}{February 15, 2019}
\newcommand{\hmwkClass}{Algorithms and Complexity}
\newcommand{\hmwkClassTime}{}
\newcommand{\hmwkClassInstructor}{Professor Bradford}
\newcommand{\hmwkAuthorName}{\textbf{Patrik Sokolowski} }

%
% Title Page
%

\title{
    \vspace{2in}
    \textmd{\textbf{\hmwkClass:\ \hmwkTitle}}\\
    \normalsize\vspace{0.1in}\small{Due\ on\ \hmwkDueDate\ at 3:10pm}\\
    \vspace{0.1in}\large{\textit{\hmwkClassInstructor\ \hmwkClassTime}}
    \vspace{3in}
}

\author{\hmwkAuthorName}
\date{}

\renewcommand{\part}[1]{\textbf{\large Part \Alph{partCounter}}\stepcounter{partCounter}\\}

%
% Various Helper Commands
%

% Useful for algorithms
\newcommand{\alg}[1]{\textsc{\bfseries \footnotesize #1}}

% For derivatives
\newcommand{\deriv}[1]{\frac{\mathrm{d}}{\mathrm{d}x} (#1)}

% For partial derivatives
\newcommand{\pderiv}[2]{\frac{\partial}{\partial #1} (#2)}

% Integral dx
\newcommand{\dx}{\mathrm{d}x}

% Alias for the Solution section header
\newcommand{\solution}{\textbf{\large Solution}}

% Probability commands: Expectation, Variance, Covariance, Bias
\newcommand{\E}{\mathrm{E}}
\newcommand{\Var}{\mathrm{Var}}
\newcommand{\Cov}{\mathrm{Cov}}
\newcommand{\Bias}{\mathrm{Bias}}

\begin{document}

\maketitle

\pagebreak

\begin{homeworkProblem}
  Chapter 3, Exercise 1.
\vspace{5mm}\\

Consider the directed acyclic graph G in Figure 3.10. How many topological orderings does it have? \\
  \begin{center}
  \includegraphics{figure.PNG}\\
  Picture Provided by Textbook (Algorithm Design by John Kleinberg) \\

\vspace{5mm}\\

6 Topological orderings:\\
\vspace{4mm}\\
a,b,c,d,e,f\\
a,b,d,c,e,f\\
a,b,d,e,c,f\\
a,d,e,b,c,f\\
a,d,b,c,e,f\\
a,d,b,e,c,f\\
(Topological Sort)\\    
   \end{center}

\end{homeworkProblem}



%
%
\pagebreak
%
%

%
%
%
\begin{homeworkProblem}
  Chapter 3, Exercise 2.
\vspace{5mm}\\

    Give an algorithm to detect whether a given undirected graph contains a cycle. If has a cycle it should output one. Running time shoud be O(m+n), n nodes and m edges.\\
\vspace{5mm}\\

Run BFS algorithm on an arbitrary node s, resulting in a list/tree T.\\
.\hspace{1cm} If every edge of L (our list/tree) appears in the BFS tree, then L = T.\\
.\hspace{2cm} so L is a list/tree which has no cycles.\\
.\hspace{1cm} Otherwise there is an edge e=(v,w) that belongs to L but not T.\\
.\hspace{2cm} Considering the least common ancestor u of v and w in T.\\
.\hspace{3cm} We get the cycle from the edge e, together with u-v and v-w paths in T.\\
\vspace{5mm}

We can start with either BFS or DFS which would change our algorithm slightly depending on which is chosen.


\end{homeworkProblem}


%
%
\pagebreak
%
%

\begin{homeworkProblem}
  Chapter 3, Exercise 3.

Extend the topological ordering algorithm so that, given an input directed graph G, it outputs one of two things: (a) a topological ordering, thus establishing the G is a DAG, or (b) a cycle in G, thus esablishing that G is not a DAG. Running time should equal O(m+n) with n nodes and m edges.\\
\vspace{5mm}\\

  \begin{center}
  \includegraphics{figure2.PNG}\\
  Picture Provided by Textbook (Algorithm Design by John Kleinberg) \\
  \end{center}
\vspace{5mm}\\


Run BFS algorithm on an arbitrary node s, resulting in a list/tree T.\\
.\hspace{1cm} If every edge of G (our list/tree) appears in the BFS tree, then G = T.\\
.\hspace{2cm} so G is a list/tree which has no cycles.\\
.\hspace{3cm} Return topological ordering of G (running topological algorithm above) establishing G is not a DAG.\\
.\hspace{1cm} Otherwise there is an edge e=(v,w) that belongs to G but not T.\\
.\hspace{2cm} Considering the least common ancestor u of v and w in T.\\
.\hspace{3cm} We get the cycle from the edge e, together with u-v and v-w paths in T.\\
.\hspace{3cm} Return that cycle, establishing G is not a DAG.\\
\vspace{5mm}


\end{homeworkProblem}


%
%
\pagebreak
%
%


\begin{homeworkProblem}
  Chapter 3, Exercise 5.

SHow by the induction that in any binary tree the number of nodes with two children is exactly less than the number of leaves.\\
\vspace{5mm}\\

\textbf{Proof by Induction:}\\
\vspace{5mm}\\

.\hspace{1cm} \textbf{Base Case:} root node with 0, 1, or 2 leaves\\
.\hspace{2cm} Consider the root has no children, then it is a leaf and the tree has 1 leaf with no node with 2 children.\\
.\hspace{2cm} Consider the root has a single leaf, then again the tree has 1 leaf anf no nodes with 2 children.\\
.\hspace{2cm} Consider the root has 2 children that are leafs, then the tree has 1 node with 2 children and 2 leafs. \\
\vspace{1mm}\\

.\hspace{1cm} \textbf{Inductive hypothesis:} T is a binary tree with 'n' nodes.\\
.\hspace{2cm} The number of nodes with 2 children is exactly 1 less than the number of leaves.\\
\vspace{1mm}\\

.\hspace{1cm} \textbf{Inductive step:} consider both cases:\\
.\hspace{2cm} A parent with only 1 child\\
.\hspace{2cm} A parent that is a leaf\\
.\hspace{1.5cm} The number of leaves and nodes with 2 children both increment by 1.\\
.\hspace{1.5cm} Proving I.H. to be true.\\

\end{homeworkProblem}

%
%
\pagebreak
%
%


\begin{homeworkProblem}
  Chapter 3, Exercise 8.\\
\vspace{5mm}
\begin{center}
claim: There exists a positive natural number c so that for all connected graphs G, it is the case that:\\
$\frac{diam(G)}{apd(G)}$  \leq c. \\
\vspace{5mm}\\
\textbf{false!}\\
\vspace{5mm}\\


Consider constant k,\\
Take a path to k-1 nodes, u_1 , u_2 , ... , u_{n-1}  \\
Attach\,\, additional\,\, n-k+1\,\, nodes\,\, (attaching\,\, an\,\, edge\,\, to\,\, u)\\
.\hspace{1cm} v_1 , v_2 , ... , v_{n-k+1} \\
diameter\,\, of\,\, G\,\, now\,\, equals\,\, dist( v_1 , v_{k-1} ) & = & k \\
pairing\,\, both\,\, lists\,\, makes\,\, their\,\, distance\,\, $\leq$ k.\\
If\,\, we\,\, choose\,\, n-1\,\, $>$ 2k^2 \,\,we\,\, get\,\, apd(G) \,\,lower than\,\, 3. \\
Given\,\, k\,\, $>$ 3c :\\
$\frac{diam(G)}{apd(G)}$  & $>$ & $\frac{3c}{3}$ & = & c\\
Which\,\, disproves\,\, the\,\,
 claim.\\

\end{center}

\end{homeworkProblem}

%
%
\pagebreak
%
%


\end{document}
